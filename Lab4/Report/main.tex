\documentclass{labReport}
\urlstyle{same}

\let\verbatim\undefined 
\let\verbatimend\undefined 
\lstnewenvironment{verbatim}{\lstset{breaklines,basicstyle=\ttfamily}}{}
\usepackage{lipsum}

\title{Lab 4: Divide and Conquer Algorithms}
\author{Adam Haile, Aiden Miller, Leigh Goetsch}
\prof{Dr. Berisha}
\className{Algorithms \& Adv. Data Struct.}
\classCode{CSC 3310}
\semester{Fall 2024}
\submissionDate{11/08/2024}
\labWeek{8}
\laboratoryDate{10/25/2024}

\begin{document}
\maketitle

\section*{Learning Outcomes}
\begin{itemize}
    \item Design an algorithm for a given computational problem statement
    \item Justify the correctness of an algorithm
    \item Perform asymptotic complexity analysis of the run time of an algorithm
    \item Generate test cases for an algorithm
    \item Correctly implement an algorithm from pseudocode
    \item Design and execute benchmarks for an algorithm
\end{itemize}

% if you want a TOC:
% \tableofcontents

\newpage
% if you want to use multicols:
\begin{multicols*}{2}
\raggedcolumns



\section{Approach}
% A paragraph describing the approach that you used to solve the problem. Provide at least 2 illustrations that explain the approach

\section{Pseudocode}
% High-level pseudocode for an algorithm that uses that rule to solve the computational problem for any input

\section{Algorithm Justification}
% 3. Provide an explanation and justification for why your algorithm is correct (1-3 paragraphs)



\section{Test Cases}
% A table of your test cases, the answers you expect, and the answers returned by running your implementation of the algorithm

\section{Recurrance Relation}
% Derive a recurrence relation describing the run time in terms of the number of points n. (Hint: assume that the random pivot divides the elements in half each time.
% Solve the recurrence relation to get a run time in terms of n in asymptotic notation

\section{Benchmarking}
% Benchmark your implementation versus an approach that sorts the numbers and picks the element at index k – 1. You should include a table and graph from benchmarking different lists with different sizes of numbers. The benchmarks should support your theoretically-derived run time and provide evidence that the run time of your algorithm grows more slowly than the sorting approach.

\end{multicols*}

% add appendix
% 7. Attach all your source code and test cases in an appendix.



\end{document}
