\documentclass{labReport}
\urlstyle{same}

\let\verbatim\undefined 
\let\verbatimend\undefined 
\lstnewenvironment{verbatim}{\lstset{breaklines,basicstyle=\ttfamily}}{}
\usepackage{lipsum}

\title{Lab 4: Divide and Conquer Algorithms}
\author{Adam Haile, Aiden Miller, Leigh Goetsch}
\prof{Dr. Berisha}
\className{Algorithms \& Adv. Data Struct.}
\classCode{CSC 3310}
\semester{Fall 2024}
\submissionDate{11/08/2024}
\labWeek{8}
\laboratoryDate{10/25/2024}

\begin{document}
\maketitle

\section*{Learning Outcomes}
\begin{itemize}
    \item Read and interpret a written description of an algorithm
    \item Correctly implement an algorithm from pseudocode
    \item Generate test cases for the implementation
    \item Design and execute benchmarks for an algorithm
\end{itemize}

% if you want a TOC:
% \tableofcontents

\newpage
% if you want to use multicols:
% \begin{multicols*}{2}
% \raggedcolumns

\section{Paper Review}
\begin{enumerate}
    \item How do scapegoat trees compare with Red-Black, AVL, and splay trees? Why might you prefer to use or not use a scapegoat tree?\\

    \item What does it mean for a node to be weight balanced? What does it mean for a tree to be weight-balanced? Draw some examples and calculate their weight balances.\\

    \item What is the interpretation of the $\alpha$ parameter?\\

    \item What are the conditions for triggering a rebuild of a subtree (during inserts) or the entire tree (during deletes)?\\

\end{enumerate}

\section{Implementation}


2. Implement a scapegoat tree that supports insert, size, delete, and contains operations. The tree
should additionally support a toList() operation that generates a list from an in-order traversal.
Add a counter variable to keep track of the number of times the rebuild operation is performed.
You do not need to implement the logarithmic space rebuild algorithm described in 6.1 and 6.2 –
use the straightforward approach.
3. Write unit tests that involve the insert, remove, size, contains, and toList() operations.
4. Benchmark the insert, delete, and contains operations of your implementation on data sets of
different sizes. Create tables and plots that include both run times and the number of times the
rebuild operation was performed.
5. Analyze and interpret the benchmark results to determine if the run time of your implementation
is consistent with the theoretical analysis.

% \end{multicols*}

% add appendix
% 7. Attach all your source code and test cases in an appendix.



\end{document}
